\chapter*{Preface}\label{ch_preface}
\addcontentsline{toc}{chapter}{Preface}

This book was written by a group of researchers associated with UC Merced's Cognitive and Information Sciences Program (\url{http://cogsci.ucmerced.edu/}) to support learning about neural networks in a visual and interactive way. It is intended to be used in conjunction with Simbrain (\url{http://www.simbrain.net}), a free open source software package that makes it easy to build neural network simulations.\footnote{For more information see the online documentation at \url{http://www.simbrain.net/Documentation/v3/SimbrainDocs.html}, the Simbrain YouTube channel or search \#Simbrain on X (sort by ``latest'').} The philosophy behind the book is that it is possible to learn about neural networks even with minimal mathematical background, and that this is facilitated by the use of a visual simulation environment like Simbrain.

Though little mathematical background is assumed, there is a lot of mathematical detail in the book. Most of these details are included in lengthy footnotes. We have not shied away from heavy use of footnotes, since they provide a convenient way of providing additional layers of information independently from the main text. 

Currently the book focuses on neural networks specifically in cognitive science, and to a lesser extent neuroscience. Of course today neural networks are best known as a tool in machine learning, and in particular \emph{deep learning}. The book does provide some background relevant to machine learning uses of neural networks, but that is not its current focus. Given the modular nature of this book, it may develop in a way that encompasses these uses of neural networks, but it does not do so at present.

Several other sources written in the same spirit as this book should be mentioned: Randall O'Reilly and Yuko Munakata's \emph{Computational Cognitive Neuroscience} book, and associated materials, which are based on the free, open source Emergent simulation platform\footnote{\url{https://compcogneuro.org/}}, as well as several sources that provide more guidance on deep learning and machine learning, with the assistance of interactive tools and visualizations, some of which run directly in the browser.\footnote{See \url{http://neuralnetworksanddeeplearning.com/} and the articles at \url{https://distill.pub/}, as well as \url{https://playground.tensorflow.org/}. Also see  \url{https://www.tensorflow.org/tensorboard/get_started} and Jay McClelland's Matlab-based course: \url{https://web.stanford.edu/group/pdplab/pdphandbook/}}

This book is meant to be improved, corrected, and expanded on a regular basis, and hopefully, remixed and remastered by others.\footnote{Many issues and plans for improvement are included as comments in the latex documents, which you are welcome to peruse.} If you fix or improve something, please submit a pull request, and if you have suggestions, post an issue on the github repository, which is here: \url{https://github.com/simbrain/NeuralNetworksCogSciBook}. The plan is to have regular releases each year. Hence, the year-based versioning, e.g. version 2022.1, 2022.2, etc.

To support this flexibility, custom scripts and \LaTeX~commands are provided. The most complete version of the book (the ``master document'') is hosted on the github repository. Those chapters can be combined and remixed in your own ``container'' documents that only contain the information you need for a particular use (\eg for a particular class you are teaching). You can fork the repository and  create your own container documents containing whichever chapters you like, including new material of your own, or adaptations of existing chapters. Guidelines for assembling your own container documents, and for producing new chapters (which we hope you will share with us!), are in the readme document of the github repo, which can be found just by scrolling to the bottom of \url{https://github.com/simbrain/NeuralNetworksCogSciBook}.

All glossary items are listed in \textbf{bold face}. For any bold faced glossary item there should be a corresponding entry in the glossary in the back of the book.

Chapter authors are listed in the order of their contribution to that chapter. Authors listed on the front cover of a container document are ordered by the weighted sum of their contributions to the chapters in that container document. Author orderings are produced using a python script included in the repository.

When references have a ``*'' symbol attached to them, it means that they refer to a chapter, section, or figure in the master document but not that container document. The master document is a kind of global container document hosted at the main github repository, that contains all chapters known to the original team.

The first version of this book was written by Jeff Yoshimi, as was the infrastructure to support it. Graphics support was provided by Pamela Payne, Elizabeth Reagh, and Soraya Boza (credits for individual figures are listed at the end of the document). David Cuesta, Eric Schwitzgebel, Eric Thomson reviewed several chapters in 2024, and additional feedback was provided by Shervin Nosrati, Kate Totter, Julia Ton, and Matthew Lloyd. Starting in 2024 we began to use AI to help with some tasks, like formatting latex and suggesting examples and finding typos. All AI suggestions are thoroughly reviewed and revised by human authors.  Sergio Ponce de Leon reviewed several chapters in 2022 and 2024. Liza Oh reviewed several chapters in Fall 2021. Tim Meyer helped review and edit the Spring 2017 and Fall 2017 versions of the manuscript.  Sharai Wilson provided a great deal of help with the manuscript in Summer 2017. Every time the course is taught students and teaching assistants provide valuable feedback, going back to 2006 (Spring term of the year UC Merced opened, and the first time an earlier version of this text was used). Ricardo Velasco helped with many aspects of producing the first versions of this text in the 2000s.

As noted above, the book is closely tied to a separate open source project, Simbrain. Simbrain credits are here: \url{http://simbrain.net/SimbrainCredits.html}. 

This work is licensed under the Creative Commons Attribution 4.0 Attribution-ShareAlike 
CC BY-SA  License. To view a copy of this license, visit \url{https://creativecommons.org/licenses/by-sa/4.0/}. As noted in the description of the license, this allows the content here to be extended and remixed, but assumes that in such a case changes be noted ``but not in any way that suggests the licensor endorses you or your use.''  

\begin{figure}[h]
\includegraphics[scale=.7]{./images/CC_License.png}
\end{figure}
