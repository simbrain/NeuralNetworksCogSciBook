\chapter{Supervised Learning}\label{ch_supervised}
\chapterauthor{Jeff Yoshimi}

With supervised learning, weights are changed using an explicit representation of how we want the network to behave. Input vectors are associated with targets or labels in a labeled dataset (see chapter \extref{ch_data_science}). We say, ``if you see this pattern, produce this other pattern.''  This is sometimes called ``learning with a teacher.''  We saw in chapter \extref{ch_unsupervised} that Hebbian learning can do something like supervised learning, where we expose it to input / output pairs and update weights using the Hebb rule. Unfortunately, that method is unstable. The weights tend to explode to extreme values. So we need something more adaptive and robust: a way to get the weights to go up and down and settle on just the right values, so that our network gets as close as possible to doing what we want. That's what supervised learning algorithms provide for us.

In this chapter we focus on general features of supervised learning in feed-forward networks, developing a toolkit of conceptual techniques and visualization methods. We will need to think clearly about labelled datasets, distinguish classification from regression tasks, learn how to visualize these tasks, and discuss how to compute a metric of how well our network is doing at a given time (``error''). Finally, we will need to think about error \emph{reduction} in a visual way, as downward motion on an error surface using the method of ``gradient descent'', which is basically the dynamical systems idea from chapter \extref{ch_dst} of finding attracting fixed points, but this time in weight space rather than activation space. 

In chapter \extref{ch_supervised_ff} we cover some of the main classes of algorithm in supervised learning, and the implications of these ideas for cognitive science. In chapter \extref{ch_supervised_recurrent} we discuss how supervised learning methods can be used to train recurrent networks, and how these trained recurrent networks have illustrated ideas in cognitive science. In both cases, we will see that internal representations are often learned by these networks that seem to be similar to those humans use in processing language, recognizing faces, and other tasks. 

% Use function approximation language from Deep Learning?
% Wiki ``Supervised learning is the machine learning task of inferring a function from labeled training data''

\section{Labeled datasets}\label{supervised_datasets}

With supervised learning, we tell the network what we want it to do. There is a teacher or trainer. Recall from chapter \extref{ch_data_science} that a \emph{labeled dataset} for a supervised learning task consists of a pair of datasets: an input dataset $I$ and a target dataset $T$.\footnote{As discussed in chapter \extref{ch_data_science}, the term `label' is sometimes reserved just for classification tasks, but is sometimes generalized, as here, to apply to regression tasks as well.} We can represent this as a pair $(I,T)$. Both $I$ and $T$ contain the same number of rows. We label the input vectors in $I$ by ``$\textbf{x}_r$''  and target vectors in $T$ by ``$\textbf{t}_r$''. We call $r$ the ``row index''.\footnote{In Simbrain, if you double click on supervised learning networks, like LMS or Backprop networks, you will notice there are separate tabs for these two tables. See \url{http://www.simbrain.net/Documentation/docs/Pages/Network/training/trainingDialog.html}. }   
% Mention y_r as well.
% Integrate: Since the rows of the tables are vectors (cf. chapter \extref{ch_linear_algebra}) they are represented here by bold faced symbols, like $\mathbf{t}_r$ and $\mathbf{y}_r $. 

We will say that a labeled dataset $(I,T)$ is \emph{compatible} with a feed-forward neural network if the number of columns in $I$ is the same as the number of input nodes in the network and the number of columns in $T$ is the same as the number of output nodes in the network. A labeled dataset $(I,T)$ can be used to train any network compatible with it. Examples of labeled datasets and compatible networks are shown in figure \ref{tables_nets}. 
 
For a labeled dataset to be compatible with a network, the only requirements concern the input and output nodes. However, note that the network could have any number of hidden layers of any size. In figure \ref{supervised_datasets}, each network could have a single layer of hidden units, multiple layers of hidden units, or no hidden units at all, and still be matched to its corresponding network.

A labeled dataset $(I,T)$ can be thought of as a \emph{contract} for a pattern association task: we'd like to train a network to come as close possible to implementing the input-output associations described by our labeled dataset. We'd like a network to produce an output vector as close to $T$ for each  input vector in $I$.

\begin{figure}[h]
\centering
\includegraphics[scale=.4]{./images/TablesAndNets.png}
\caption[Jeff Yoshimi.]{Some labeled datasets and the types of neural network topologies those training sets could be used on. Each dataset contains an input dataset $I$ and a target dataset $T$. In each case $I$ and $T$ both have $R=4$ rows. In each case the $I$ has as many columns as its paired network has input nodes and $T$ has as many columns as its paired network has output nodes. A classification task is shown in the middle (binary valued targets) and regression tasks are shown on the left and right (real-valued targets).}
\label{tables_nets}
\end{figure}

\section{Supervised Learning: A First Intuitive Pass}\label{SupervisedFirstPass}
% Refer to "schema for supervised learning"?
% Fold in some of the vector valued function stuff that was removed from the learning chapter?

 In this chapter we focus on feed-forward networks $N$, which can be thought of implementing vector valued functions. A labeled dataset $(I,T)$ is essentially a specification for a vector-valued function we'd like our compatible network $N$ to implement. Given an input vector $\mathbf{x}_r$ in the input dataset $I$, we want the network to produce an output vector $\mathbf{y}_r = N(\mathbf{x}_r)$ in the target dataset $T$. 

% A labeled dataset compatible with $N$, containing an input dataset and a target dataset. The rows of the input dataset are $\mathbf{i}_i$, the the rows of the target dataset are $\mathbf{t}_i$, and the outputs of $N$ (the output dataset) are $\mathbf{y}_i$. Note that if we refer to $\mathbf{i}_i$ and $\mathbf{t}_i$, for example, we are referring to the  "corresponding" rows of the two datasets.

The way we do this with supervised learning is by using algorithms that modify the \glossary{parameter}s  $p_1,\dots,p_n$ of $N$, primarily the weight strengths and biases of the nodes. We start out with a network all of whose parameters $p_i$ have been initialized to random values.\footnote{When I refer to "randomizing" a set of values, we mean setting them to values generated by a probability distribution, \eg a uniform or a Gaussian distribution.}  It's like making a network in Simbrain and pressing the \emph{w} then \emph{r} buttons, which selects all the weights and randomizes them. Recall from Chap. \extref{ch_dst} that parameters are variables associated with a dynamical system that  are fixed when the system is run but can be changed between runs.\footnote{When neural networks are not being trained, weights are thus parameters in the dynamical systems sense. However, when training begins, the parameters become state variables of a new dynamical system on the weight space of the network (more on this in section \ref{sect_gradient_descent}).} Our goal is to set the parameters of the network $N$ so that it implements a vector-valued function that reproduces the associations coded in the labeled dataset as closely as possible: so that $N(\mathbf{x}_r) \approx \mathbf{t}_r$ for each row $r$ of the labeled dataset.

Because we start with random values for the parameters of $N$, it will generally not do well at first. It's outputs won't  initially match target values. We then use a learning algorithm (several are covered later in this chapter) to incrementally update the parameters. If all goes well, the network should begin to behave in accordance with the specifications of the labeled dataset.

Below we refer to ``row errors''  and ``overall error'', which are discussed in greater detail in Sect. \ref{sect_error}. Roughly speaking row errors says how far away the outputs produced by an input vector are from the target values for that input, and overall error combines the row errors.

% Language of ``prediction of model'' for outputs
Here is a schematic which covers most forms of supervised learning, given a network $N$ and a compatible labeled dataset $(I,T)$:
 \begin{enumerate}
\item Randomize network $N$'s parameters $p_1,\dots,p_n$.
\item For each row  $\mathbf{x}_r$  of  the input dataset $I$:
\begin{enumerate}
\item Set the input-layer activations of $N$ to $\mathbf{x}_r$.
\item Compute $N$'s resulting output vector $\mathbf{y}_r$. 
\item Compute row-errors by comparing the targets $\mathbf{t}_r$ with the outputs $\mathbf{y}_r$.
\item Update $p_1,\dots, p_n$ with the goal of reducing row errors (so that outputs are closer to targets). %[note variations; this is not always done at this stage]
\end{enumerate}
\item Repeat step 2 until overall error is sufficiently low.
\end{enumerate}
	
% Mention stochastic gradient descent and mini-batch as variations on this. 
% Introduce the term epoch to refer to one pass through the entire dataset. Bold face these some of these terms for the glossary.
% Discuss testing dataset, and return to the discussion of generalization. This also matches the tensorflow demo.

We will see that this can be visualized in geometric way, as gradient descent to a low point on an ``error hypersurface.''

\section{Classification and Regression}\label{classificationRegression}

For feed-forward networks trained using supervised learning, an important distinction can be made between classification and regression tasks. At a first pass, classification tasks associate inputs with categories, and regression tasks associate inputs with numbers.\footnote{These concepts (especially classification) can also be applied to unsupervised learning. Recall from chapter \extref{ch_unsupervised} the discussion of competitive networks and self organizing maps which learn to classify inputs into distinct categories without a teacher. The distinction also applies to recurrent networks, which can learn to classify  dynamic inputs, for example, or to produce dynamic real valued outputs.} 

In a \glossary{classification task}, the network sorts inputs into categories. The inputs might be the height and weight of a person, and the output will be a prediction about whether that person is a child or adult. Or the network could take car data as input, and predict whether the car is a sports car or economy car. Notice that in both cases the outputs are categorical, they say which of $n$ categories something falls in. This is one-of-$k$ or one-hot encoding, discussed in chapter \extref{ch_data_science}. One node for each category, and the one that's on corresponds to the category being classified. 
% note threshold activation function historically, and softmax for cases where outputs are probabilities

So, if the output vectors in a training task are binary vectors where a  $1$ represents category membership, then the network is a classifier. The 3 object detector (figure \extref{3ObjectClassifier}) is a classifier, which classifies smell inputs into one of three categories: Fish, Gouda, and Swiss. figure \ref{F:letter_classify} shows an example of a feed-forward network that classifies pixel patterns as one of 26 letters.
  
\begin{figure}[h]
\centering
\includegraphics[scale=.4]{./images/letterClassification.png}
\caption[Simbrain screenshot.]{An example of classification. A feed-forward network trained via supervised learning to classify letter inputs as specific letters.}
\label{F:letter_classify}
\end{figure}

In a \glossary{regression task} a  network trained has real valued target values, not binary values. This is simply the more general case of estimating a vector-valued function, where there are no constraints on how we interpret the outputs. If we train a network to predict the speed of a car (quarter-mile time) based on its fuel efficiency (in mpg), engine size, and how many cylinders it has, we have a regression problem. We are not classifying cars into types, but predicting a numerical quantity about cars: their speed.

In the case of supervised learning it will be important to return to the distinction between classification and regression tasks. Recall that in a \glossary{pattern classification} task we associate input vectors with category labels, which are usually coded with 0's and 1's (e.g. with a one-hot encoding). The input might be the height and weight of a person, and the output will be boy or girl. Or we could take the car data as input, and then try to have it say whether the car is a sports car or economy car as output. With a \textbf{regression} task, by contrast, the target values are  real valued, not binary valued. This is simply the more general case of estimating a vector-valued function, where there are no constraints on how we interpret the outputs. If we train a network to predict the exact mileage and cc's of an engine based on how many cylinders it has, we have a regression problem. We are not classifying cars in to types, but predicting some numerical quantities about them. 
% (See Duda-Stork 1-d examples)
% Emphasize that regression is closer to the concept of function approximation . Classification is a different conceptual game, and a bit easier to start with. Could use this as the running thread.

% Improve. Blue dots. Perhaps make 2 node network vertical.
\begin{figure}[h]
\centering
\raisebox{-0.5\height}{\includegraphics[scale=.35]{./images/linearRegressionTable.png}}
\hspace*{.4in}
\raisebox{-0.5\height}{\includegraphics[scale=.4]{./images/linearRegression.png}}
\hspace*{.4in}
\raisebox{-0.5\height}{\includegraphics[scale=.3]{./images/2Node_Network.png}}
\caption[Jeff Yoshimi.]{An example of regression. (Left) The data used to train the network. Hours of study vs. score on the SAT. (Middle) A plot of the data with a regression line. These are points in an input-target space, as discussed in chapter \extref{ch_supervised}. This line can be used to predict how well someone will do based on how much they study. (Right) A simple 2-node network that could implement this regression solution. Enter hours studied in the input node, and it should display a  predicted SAT score in the output node.}
\label{F:linearRegression}
\end{figure}

The term ``regression'' comes from the statistical technique of linear regression. In fact, neural networks provide a nice way to understand what linear regression is. To understand this, consider a classic example of linear regression: predicting how well someone will do on a test based on how many hours they study. It turns out, these data tend to be nicely correlated, as in figure \ref{F:linearRegression}. The more you study, the higher your score is likely to be. We can fit a line to this data using standard statistical techniques. But a neural network--like the one shown in the right panel of the figure--can also do it.\footnote{Note that in this example the data points have not been rescaled. Rescaling is often important, but not always necessary. Also note that the slope of the line corresponds to the weight, and the intercept is the bias on the output node. Think about it!  In computing weighted inputs $n_k$ for output node $k$, when there is just one input, we are just computing a simple linear function.} That network can be trained to predict SAT scores based on hours studied. Look how simple it is!  That's all linear regression really does: it gives us a network that we can use to predict something with: one input, one output. 

Of course we can get more complex. We can have multiple inputs. We can  predict how tall a tree will be based on its age, average rainfall where it is planted, and concentrations of chemicals in its soil. In that case we have multiple inputs predicting one output. In statistics this is called \emph{multiple regression}, but you can see that it's just a matter of having  a many-to-one network where we estimate the values of the weights and biases. When we have multiple outputs we just repeatedly use this technique on each output node. One output node predicts the height of the tree, another predicts how long it will live, etc. That is called \emph{multi-variate multiple regression}. Thus networks with many inputs and many outputs can  be understood as performing regression tasks.

As a simple procedure for deciding whether a task is a classification or regression task, look at the target data. If the target data represent categories, e.g. a one-hot encoding, it is probably a classification task. If they are real-valued or otherwise numerical data, then it is probably a regression task. Even more simply: classification tasks typically involve binary or discrete valued targets, while regression tasks typically  involve real-valued, numerical targets. 

\section{Visualizing Classification and Regression}\label{visualizeRegressionClassification}

% In all classification plots plot classes of two points in a different color. Then it coheres with the python example better.
% Could show how the pictures are related. Show a function surface and then a threshold and their intersection is the decision surface. Something like that.

% An outstanding source on these issues: http://colah.github.io/posts/2014-03-NN-Manifolds-Topology/
% Note that classification tasks don't always use linear decision boundaries!  E.g. Bishop p. 180.. All the m ore important because backprop finds non-linear boundaries!! 
% Useful text on regression vs. classification: Bishop p. 194. 
It is important in supervised learning to be able to conceptualize problems in terms of a set of graphical ideas. They are familiar ideas, and not too hard, but they confusingly overlap, so we must be careful and systematic about understanding them. You will see many charts that look quite similar. We will only be able to visualize what's going on directly for very small networks, but we can use these ideas to generalize to higher dimensions, which will give us a conceptual template for understanding more complex cases. This theme of visualizing ideas directly in small networks and then extending them to higher dimensions is often useful, as we will see.
% Above: consider replacing visualize with conceptualize. And/or clarify that visualize for 1-3 dims and conceptualize after that.

Our first graphical task is to understand regression and classification in a visual way:
\begin{description}
\item[Classification:] The goal of a classification task is to create a model that correctly classifies inputs into a finite set of categories. Target values are binary; an input is either in a given category or not. Classification involves creating decision boundaries between inputs for the different categories. 
\item[Regression:] The goal of a regression task is to create a network that produces outputs as close as possible to a set of datapoints. Targets are real-valued. We can conceptualize regression as fitting a hypersurface (a generalization of lines, planes, and other surfaces to arbitrary dimensions)\footnote{\url{https://en.wikipedia.org/wiki/Hyperplane}.} as close to a cloud of data points as possible.\footnote{Regression models need not be ``flat'' like this. When sigmoidal node are used, for example, they will be more like wavy curves and surfaces. But we will not consider those cases here.}
\end{description}
% Threshold vs. linear or sigmoidal activation functions

% Fix all the graphics. Datapoints in blue, models in black. Next two figures.
\begin{figure}[h]
\centering
\includegraphics[width=0.9\textwidth]{images/visualizeClassification.png}
\caption[Jeff Yoshimi.]{A classification task for a 1-1 network (Left) and a 2-1 network (Right). Both networks use threshold activation functions on the output nodes. Points in the input space are shown in  blue. The decision boundaries between points classified as 0 or 1 are shown in gray. On the left, the decision boundary is a point shown as a small vertical hatchmark. On the right, the decision boundary is a diagonal line. The decision boundaries partition the input spaces into two decision regions, corresponding to outputs of 0 or 1.}
\label{visualize_classification}
\end{figure}

Classification tasks involve dividing an input space using decision boundaries (one boundary for each output node). Figure \ref{visualize_classification} shows classification tasks for 1-1 and 2-1 networks. In each case the output node has a threshold activation function, and the network is being trained to classify inputs (so that target data are binary). Here we only focus on the input space. When the output node turns on (weighted inputs above threshold), the input is in class 1. When the output node is off the input is in class 0. On the left, the input space is 1-dimensional, since there is one input node. The threshold function divides the 1-dimensional input space into 2 \glossary{decision regions},  labeled ``class 0'' and ``class 1'', via a \glossary{decision boundary}, in this case a 0-d point (represented by a hatchmark in the graph).\footnote{For more on decisions regions and boundaries, see \url{https://www.cs.princeton.edu/courses/archive/fall08/cos436/Duda/PR_simp/bndrys.htm}.} On the right, the input space is 2-dimensional, and the decision boundary is 1-dimensional. The line again partitions the input space into two decision regions, for ``class 0'' and ``class 1.''  

These ideas generalize to higher dimensions. The \emph{input space} will in general have as many dimensions as there are input nodes. The small networks in Figs. \ref{visualize_classification} and \ref{visualize_regression}  have 1 and 2-d input spaces. A network with 5000 input nodes has a 5000-dimensional input space. We already saw this idea in chapter \extref{ch_unsupervised}. Note that for visualizing an input space dimensionality reduction is often helpful. The \emph{decision boundary} in the input space of a network, relative to a classification problem, has as many dimensions as the number of input nodes minus 1. In the 5000-input node case, with one output node, the decision boundary is a 4999-dimensional hyperplane that divides the input space into two decision regions.

To summarize, for classification tasks we have:
\begin{description}
\item[Input space:] the vector space corresponding to the input nodes of a network. It has as many dimensions as there are input nodes. In the cases shown in figure \ref{visualize_classification} the input spaces are 1 and 2 dimensional.
\item[Decision boundary:] a point, line, or surface separating the input space into decision regions corresponding to distinct categories. The boundary has as many dimensions as the number of input nodes minus one. In the cases shown in figure \ref{visualize_classification} the decision boundaries are $1-1=0$ dimensional (a point) and $2-1=1$-dimensional (a line). Note that we are only considering the case of one output node here, which divides the input space up into two regions. With more output nodes, more decision boundaries are added.
\end{description}
		
Regression involves fitting a function to a cloud of points. In the simple case shown in figure \ref{visualize_regression}, left,  we just have 1 input and 1 output. This is like graphing a function in high school algebra. The graph of the function is 1-dimensional, \ie a line.\footnote{In mathematics, even a curvy line is 1-dimensional, and even a wavy plane is 2-dimensional, even if they can only be visualized in a higher dimensional input-target space. Similarly for higher dimensions.} The graph of the function is one dimensional because there is one input node. However it is shown in a 2-dimensional  input-target space, which is 2 dimensional because there is 1 input node plus 1 output  node. Input/target pairs (\ie rows of the labeled dataset) are plotted as points. Fitting the linear model, the neural network, can be thought of as turning two knobs: one for the weight (which sets the slope), and one for the bias (which sets the y-intercept). Think of trying to turn these knobs until the line (the ``model'') fits the data as best as possible. Training this kind of network is like fitting a linear regression model. Try to keep this easy-to-visualize example in mind even for much more complex networks, where there are many more knobs, and where the model being fit exists in many more dimensions.

% Take the boundaries out of the linear picture
\begin{figure}[h]
\centering
\includegraphics[scale=.4]{./images/visualizeRegression.png}
\caption[Jeff Yoshimi.]{A regression task for a 1-1 (Left) and a 2-1 network (Right) network of linear nodes. Datapoints in the input-target space are shown in  blue. The network implements a linear function from inputs to outputs, which is a line in the 1d-to-1d input-target space, and a plane in the 2d-to-1d input-target space. }
\label{visualize_regression}
\end{figure}
%In the regression case shown on the left, the output node has a linear activation function. We try to set the weight and bias of the network to fit the data. This is a \emph{graph of a function} and the space in the picture is the input-target space of that graph. 

Let's see how this works in a slightly larger network, a network with 2 input nodes and 1 output node, as in figure \ref{visualize_regression}, right. Here the graph of the function computed by the network is a 2-dimensional surface, and the input-target space of the graph is 3 dimensional (2 input nodes and 1 output node). Visualize the algorithm fitting that surface so that it's as close to the datapoints as possible. It's like fitting the line in the 1-1 network case, but now we have more knobs (the two weights and the bias of the output node) for moving the surface around. Each of the vectors in the 2-d input space is associated with a target value in the 1-d output space. The surface has been fit to the points, so that any input will be associated with outputs as close as possible to the target values. 

% The graph of function material could be more clear.
So here we have:
\begin{description}
\item[Regression hypersurface:] a generalization of the concept of a regression line to arbitrarily many dimensions. Mathematically it is the graph of an $n$-dimensional hypersurface, where $n$ is the number of input nodes. In the cases shown in figure \ref{visualize_regression} the hypersurfaces have 1 dimension (a line, for a neural network with 1 input node) and 2 dimensions (a plane, for a neural network with 2 input nodes). It is, in a sense, the ``solution'' a network learns for a regression problem. 
\item[Input-target space:] the space that the regression hypersurface lives in. this has as many dimensions as there are input nodes plus output nodes. The rows of the dataset $(I,T)$ can be conceptualized as a cloud of datapoints in this space. The goal of regression is to make the hypersurface as close to that cloud of datapoints as possible. In the cases shown in figure \ref{visualize_regression} the input-target spaces have 2 dimensions (a neural network with 1 input node and 1 target node) and 3 dimensions (a neural network with 2 input nodes and 1 target node).
\end{description}

Notice that the decision boundary in figure \ref{visualize_classification} (right) looks like the regression line in figure \ref{visualize_regression} (left). Do not confuse these! In one case we have a decision boundary for a classification task computed by a 2-1 network; in the other case we have a regression line computed by a 1-1  network.

As soon as we have networks with more than 3 nodes, our ability to visualize things starts to break down. But we have experience thinking about shapes in higher dimensions (for example via our studies of dynamics in 4-dimensional state spaces), so we should be able to do it. The regression hypersurface computed by a network has as many dimensions as there are input nodes. For networks with linear activation functions this graph can be a line, a plane, or a hyperplane. For example, if a linear network had 20 input nodes and 5 output nodes, its graph would be a 20-dimensional hyperplane in a 25 dimensional space, fit to a cloud of  points in that space.

% Connect to generalization
% Mention universal approximations here?

\section{Error}\label{sect_error}

In supervised learning we usually define an \glossary{error function}\footnote{These are also known as ``cost functions'', ``loss functions'', or ``objective functions''.} and then modify the weights and other parameters of a neural network to get the value of this error function to be as small as possible (cf. the intuitive overview in Section \ref{SupervisedFirstPass}). An error function can be thought of as a method for producing a number which describes how well a network is doing at approximating the pattern-associations encoded by a labeled dataset. We call this the \emph{overall error}, since it is associated with the \emph{entire} labeled dataset (as contrasted with \emph{row errors} which are associated with specific rows of an output-target dataset). In learning, the goal is to use mathematical techniques to change the parameters of the network so that error is as small as possible.

% Check for \textbf above,  Also make sure all the _r's are outside of them!
In practice, to compute overall error, we compute the row errors for each row of the labeled dataset $(I,T)$. We go through each row $\mathbf{x}_r$  of the input dataset, produce an output vector $\mathbf{y}_r = N(\mathbf{x}_r)$. The resulting output dataset $O$ (cf. Chap. \extref{ch_data_science}) has the same number of rows and columns as the target dataset $T$. We then go through each row $\mathbf{y}_r$ of $O$ and compare it with the corresponding row $\mathbf{t}_r$ of $T$. We often end up comparing things like $\mathbf{t}  =  (1,1)$ and $\mathbf{y}  =  (.9,.7)$ by subtracting the components of the two vectors, which gives us row errors $(1-.9,1-.7) = (.01,.03)$. 

Noice that the actual data we use to compute overall error is an output dataset together with a target dataset, \ie a pair $(O,T)$.

% Error is the flip side of \glossary{performance}. Reducing the error in a network improves its performance. We want to maximize performance by minimizing error.\footnote{In fact any optimization problem can either be thought of in terms of minimizing a cost function or, equivalently, as maximizing the negative of the cost function, which we are calling performance.} 
% Deep learning discussion of error vs. performance where they are not just symmetrical opposites. Around 268, section 8.1

% Below needs to be spelled out a bit or helped out with a pic. Distinguish pre and post-training. Pre-training off by 4 because 1 is 2 from -1, 0 is 1 from 1, etc.
Error is a pretty easy idea. If we want our network to produce the output vector $(1,1,1)$ but instead it produces $(-1,0,.2)$, well, we have kind of bad error. But if we train it and then it starts to produce the output vector $(1,1,.9)$, then we are doing much better! Intuitively, the first output is about 4 units ``off'', but the second output is about $.01$ off. That's all the basic idea involves. But saying this mathematically requires a bunch of symbols that will be intimidating to some of you. Just remember all we're ultimately doing is finding a number that says how far ``off'' the outputs are from the target values.

There are many different specific error functions that we can use  for overall error: mean squared error, cross-entropy, absolute error, etc. Knowing which to use, and when, is a more advanced topic.\footnote{One main consideration is whether you are training a network on a classification task or a regression task. SSE, which we consider here, can be used on both.} Here we are just introducing the idea. We focus on an error function called \emph{sum of squared error} or SSE. 

% "Resulting errors" is a third kind of error, component error. Think. Also see the issue in the graphics.
We will now compute SSE. Suppose we are given our network and labeled dataset, and have computed the output dataset $O$ so that we have a pair $(O,T)$, each of which has $R$ rows. To compute SSE we subtract each output vector  $\mathbf{y}_r$ in $O$ from the corresponding target vector  $\mathbf{t}_r$ in $T$,  square the components of the resulting errors, and sum them:
\begin{eqnarray*}
SSE = \sum_{r=1}^{R}(\mathbf{t}_r - \mathbf{y}_r)^2 =  \sum_{r=1}^{R}(\mathbf{t}_r - \mathbf{y}_r) \bullet (\mathbf{t}_r - \mathbf{y}_r) 
\end{eqnarray*}
$(\mathbf{t}_r - \mathbf{y}_r)^2$ is obtained by subtracting the output vector from the target vector using component-wise subtraction\footnote{This is easy to understand using an example, for example $(\mathbf{t}_1 - \mathbf{y}_1)$ where $\mathbf{t}_1 = (1,1)$ and $\mathbf{y}_1 = (2,3)$. Then  $(\mathbf{t}_1 - \mathbf{y}_1) = (1,1) - (2,3) = (1-2,1-3) = (-1,-2)$.}, then squaring the resulting vector, in the sense of taking its dot product with itself (refer to chapter \extref{ch_linear_algebra} for  a review of the dot product).

A sample computation for a network with two outputs is shown in figure \ref{error_computation2_low} (in these figures, $\sum_R$ is shorthand for $\sum_{r=1}^{R}$). Notice that SSE is low. The outputs are close to the targets. All we are really doing is subtracting a bunch of values to see how close they are, squaring them (so that the errors become positive), and then adding them up. It's an easy idea but there is a lot of notation to track. Also notice that this is a classification task (we can compute error the same way for regression).\footnote{In practice error is usually treated differently in the two cases, but in theory the same error metric can be used for both cases.}

% Change from o for outputs to y
% Add SSE = on left
\begin{figure}[h]
\centering
\includegraphics[scale=.5]{./images/ErrorComputation_2_low.png}
\caption[Jeff Yoshimi.]{Computing SSE for a network with two outputs. Error is fairly low.}
\label{error_computation2_low}
\end{figure}

\begin{figure}[h]
\centering
\includegraphics[scale=.5]{./images/ErrorComputation_2_high.png}
\caption[Jeff Yoshimi.]{Computing SSE for a network with two outputs when error is higher, as might happen when we start with random weights, or as here, weights that initially cause the output to always be 1.}
\label{error_computation2_high}
\end{figure}

\begin{figure}[h]
\centering
\includegraphics[scale=.5]{./images/ErrorComputation_1.png}
\caption[Jeff Yoshimi.]{Computing SSE for a network with one output node.}
\label{error_computation1}
\end{figure}

Low overall error (here low SSE) is something we generally see \emph{after} training a  network. When we start with an untrained network that has random weights, error will be higher, as in figure \ref{error_computation2_high}.

The same idea works for a network with 1 output node, and is much easier, since our vectors just have one component and are thus scalars. So we just subtract, square, and add!  See \ref{error_computation1}.

Try some examples of your own to get a feel for when SSE is large vs. small.

\section{Error Surfaces and Gradient Descent}\label{sect_gradient_descent}

Before we get to the main point of this section--gradient descent--we need to do a bit more visualizing. We are going to talk about \emph{another} type of graph, separate from the ones discussed in \ref{visualizeRegressionClassification}.
We will be talking about parameter spaces, or to make things easier, weight spaces (recall these have come up in chapters \extref{ch_dst} and \extref{ch_unsupervised}). On top of the weight space we will plot overall errors. That is, for each possible combination of weight values, we show what overall error (\eg SSE) would result relative to our network and training set. This  gives us an \glossary{error surface}. 

Note that error surfaces depend on the training set, and in particular targets, because this determines SSE. If you change the training set, you change the error surface. Once you fix the training set, you can now look at the error surface, which shows all possible errors for that network \emph{given} the training set.
% it can also change we change how we compute error or cost / loss more generally. 

Figure \ref{error_surfaces} shows errors surfaces for two cases that we can visualize. On the left we have an error surface for a 1-1 network where we are only adjusting a single weight. As we change that weight SSE will change too. On the right we have a 2-1 network where we are adjusting two weights. Again, as the two weights are changed so will the SSE. Notice that in each case the error surface has a single minimum point, which turns out to be convenient, but we are not always lucky enough to get such a surface.
% Convexity, when this is the case

\begin{figure}[h]
\centering
\includegraphics[scale=.5]{./images/ErrorSurfaces.png}
\caption[Jeff Yoshimi.]{(Left) error surface for a 1-to-1 network where only the weight is adjusted. (Right) Error surface for a 2-to-1 network where the two weights are adjusted. In each case the error surface also depends also the training data (the tables in figure \ref{tables_nets}). As the training data change, the error surface changes. }
\label{error_surfaces}
\end{figure}

% Make connection to graph of function?
These ideas generalize to higher dimensions, for networks with many parameters. As above, we can't visualize these cases directly, but it helps to have a visual template in mind. The error surface for a network with $n$ adjustable parameters is a surface in an $n+1$ dimensional space (the $n$ parameters plus the error term). For a network where we are adjusting three weights, we have the graph of a function from the three weights to the error, \ie a surface in a 4-dimensional space. For each possible combination of three weights we have an error (relative to the labeled dataset and the error function), and this is a surface in a 4-d space. For a network with 150 weights, we have a surface in a 151 dimensional space: each pattern of 150 weight strengths produces a specific error.

For supervised learning tasks (regression or classification), our goal is usually going to be to \emph{minimize the overall error function}. We want to find values for the parameters that make overall error, relative to the labeled dataset, as low as possible. It turns out there is a whole area of mathematics set up for problems like that. It's called \glossary{optimization}.\footnote{\url{https://en.wikipedia.org/wiki/Mathematical_optimization}.} Optimization problems involve finding either the minimum value or maximum value of a function, relative to the set of all possible inputs to the function. Here the function is the error function and the inputs are adjustable parameters, usually weight strengths and node biases. 

Optimization is useful. We often have to make decisions that involves many different variables and constraints. For example, suppose you want to buy a new laptop. You want the cost to be as low as possible, but the quality as high as possible. You need to buy it within 5 days and you really want a warranty. Often what you do is look at choices,  and when they feel  better, go in that direction,  and feel worse  tell yourself  ``no, I dislike that, look for something else''. In this way you go back and forth--generally in the direction of ``better''--and settle in on a solution. Optimization automates this kind of process. provides an automatic way of solving this kind of problem. These methods are used, for example, to determine the best ways to plant  crops to maximize yield. Supervised learning also uses optimization. A labeled dataset describes an optimization problem, a set of inputs that we want to associate with a set of targets. Mathematical optimization gives us a way to automatically update the parameters of a network so that it does the best job possible on a classification or regression task, producing outputs in response to inputs that are close as possible to their targets.
% Above add reference to generalization. We also want networks that generalize well.

\begin{figure}[h]
\centering
\includegraphics[scale=.5]{./images/GradientDescent.png}
\caption[Jeff Yoshimi.]{Error gradient on an error surface. The actual changes that happen in the weight space are shown on the horizontal axis.}
\label{gradient_descent}
\end{figure}
% Note this is not what we tend to see. In practice, different shapes. And some have tried to visualize these structures.

The main method we will discuss for minimizing the error function is  \glossary{gradient descent}.\footnote{The gradient of an error function is a vector that points in the direction of steepest increase on the error surface. The method of gradient descent involves changing a system in the direction of steepest decrease on the error surface, which is the negative of the gradient of the error function.}  In this method we start at some random point in parameter space. We begin with a network where the weights and biases and other adjustable parameters are set to random values. In the 1-to-1 network we just randomize that one weight. That puts it at a random place on the error surface over the weight space in figure \ref{error_surfaces} (left). In calculus we learn that from any point on a function, like an error surface, we can define another function, the \emph{derivative}, which shows us what the rate of change at that point is. There are a lot of details here, but basically we can attach an arrow to any point on the error surface, which says in what direction the surface is decreasing most rapidly. So we start at a random point, and then follow the arrow down from there, and then repeat the process. By iterating the algorithm in this way we ``descend the gradient.''

The process is illustrated in figure \ref{gradient_descent}. The error surface has a bowl shape. Wherever you start in the bowl, just follow the arrows down until you get to the low point. That's it!  That's how it works. 
% Convex guaranteed to have no more than one minimum, thought it might have none. You will have none or one.

We use the ``arrows'' on the error surface to derive a learning rule, which produces a \emph{dynamical system on weight space}. In chapter \extref{ch_dst} we mainly considered activation dynamics. Here we consider weight dynamics. The weight dynamics are shown in the bottom horizontal line of the figure; in that line a one-dimensional system describing how a single weight changes in order to implement a model of the training data. As noted in chapter \extref{ch_dst}, when a ``play''  button \includegraphics[scale=.5]{./images/Play.png} is pressed in Simbrain  a dynamical process is simulated. In this case we will have a tools in Simbrain for running gradient descent using a play button, and observing error reduction.\footnote{See the screenshot here: \url{http://www.simbrain.net/Documentation/docs/Pages/Network/training/trainingDialog.html}.} What we are doing is like what we did there when we looked for fixed points of a recurrent network. Here the initial conditions are random weight values, which put us at a random point in the error surface, orbits are the paths we follow in weight space (which is our state space in this case), and we usually end up  at an attracting fixed point, a weight state with low error. 

% Mention convexity below?
% Possibly clarify that global minimum is itself a local minimum (local connotes neighborhood of state space)
Unfortunately, error surfaces don't always have just a single minimum point, as in the bowl example (if they did, training networks would always be easy). They sometimes have multiple fixed points in separate basins of attraction. These are ``local minima''. Error surfaces can also have plateaus where the error will only gradually change. Both cases are shown in figure \ref{local_minima}. These can make finding the best solution difficult. Much of the mathematical theory of optimization (and  research in neural networks) is focused on dealing with these ``difficult'' error surfaces.

We usually don't have a picture of an error surface. Still, we can use a program like Simbrain to get a feel for an error surface. To do so, we start at a random point in weight space, run the algorithm, and then see what the lowest error we get is. If it does not seem so low, we try again.\footnote{It's easy to try this in Simbrain. Load up a backprop network with a training set, and train. Periodically press the random button and run again, and notice how the error changes each time. Again, this is almost exactly the same as searching for fixed points of a dynamical system,  with the state space here being a weight space.} By repeatedly doing this we get a feel for how many minima they are and how long it takes to get to them. Of course, for very  large networks, where training can take hours or days, we can't to this, but for small toy networks we can.

\begin{figure}[h]
\centering
\includegraphics[scale=.4]{./images/LocalMinima1d2d.png}
\caption[Jeff Yoshimi and Scott Hotton.]{(Left) an error surface for one weight, with two local minima and a plateau. (Right) An error surface for two weights, with two local minima.}
\label{local_minima}
\end{figure}
% Label the global, local minima and plateaus

 There is more to say here. A lot can go wrong, there are settings to adjust (like \emph{how far} you go at each iteration), etc. These details are studied in the mathematical field of optimization. Suffice it to say that one major approach to supervised learning is to design algorithms for updating the weights and other parameters of a neural network so that it minimizes an error function relative to a labeled dataset, and that gradient descent is a common way of doing this. 
 